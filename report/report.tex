\RequirePackage[l2tabu, orthodox]{nag}
\documentclass[a4paper, oneside, notitlepage]{article}

\usepackage[english]{babel}
\usepackage{fullpage}
\usepackage{fixltx2e}
\usepackage{xspace}
\usepackage{relsize}
\usepackage{url}


% font stuff
\usepackage[T1]{fontenc}
\usepackage{microtype}
\usepackage{lmodern} \normalfont
% lmodern has no smallcaps bold , so we replace it by computer-modern (closely
% related)
\DeclareFontShape{T1}{lmr}{bx}{sc} { <-> ssub * cmr/bx/sc }{}
\usepackage[scaled=0.8]{beramono} % monospace font with bold support


% nicely formatted acronyms 
%\newcommand{\Rascal}{\textsc{Rascal}\xspace}
\newcommand{\CSF}{\textsmaller{CSF}\xspace}
\newcommand{\ASFSDF}{\textsmaller{ASF\hspace{-.05em}\raisebox{.4ex}{\tiny\textbf{+}}SDF}\xspace}
\newcommand{\SDF}{\textsmaller{SDF}\xspace}
\newcommand{\ASF}{\textsmaller{ASF}\xspace}
\newcommand{\AST}{\textsmaller{AST}\xspace}
\newcommand{\CST}{\textsmaller{CST}\xspace}
\newcommand{\IDE}{\textsmaller{IDE}\xspace}
\newcommand{\LOC}{\textsmaller{LOC}\xspace}
\newcommand{\ADT}{\textsmaller{ADT}\xspace}
\newcommand{\ADTs}{\textsmaller{ADTs}\xspace}
\newcommand{\MSOS}{\textsmaller{MSOS}\xspace}
\newcommand{\JVM}{\textsmaller{JVM}\xspace}

\newcommand{\filename}[1]{\texttt{#1}\xspace}
\newcommand{\code}[1]{\texttt{#1}\xspace}
\usepackage{highlight}
\usepackage{rascal}
\usepackage{listings}

\lstset{% general command to set parameter(s) 
	basicstyle=\ttfamily\small,%
	xleftmargin=.25in,%
    xrightmargin=.25in%
}

\lstdefinelanguage{CSF}%
{%
	keywords={Glossary, Alias },% 
	moredelim=[s][\itshape]{\$}{\$},%
	lineskip={-5pt}%
}

% todo marker
\usepackage[svgnames]{xcolor}
\usepackage{marginfix}
\newcommand{\todo}[1]{%
	\marginpar{%
	\colorbox{yellow}{\parbox{0.9\marginparwidth}{%
		\raggedright \footnotesize #1}}}}

\title{Funcons in Rascal - report}
\author{Davy Landman}
\begin{document}
\maketitle

\todo{Perhaps include a short table with terms?}

\section*{Introduction}
This document will describe the translation from the \ASFSDF funcons
implementation to a \Rascal implementation.
The first section will summarize the overall architecture, the second section
will describe the manual translation to \Rascal, the third an automatic generation
approach and the last two sections will discuss the limitations and advantages
of the \Rascal implementation.


\section{The architecture}
	\subsection{\ASFSDF}
		In \ASFSDF the funcons are implemented in a two stage approach.
		The first stage is translating the \CSF~(funcon specification files) into
		\SDF definitions.
		These definitions are used to describe the semantics of a
		programming language.
		The funcons (represented as \SDF alternatives) are connected to syntax of the programming
		language using the \ASF equations.

		\paragraph{Current limitations:}
			At the time of writing, not all funcons are specified in \CSF,
			which means that not all \SDF defined funcons were generated
			from the \CSF specifications. 
			Moreover, the \CSF specifications contain the descriptions of how to
			`interpret' a funcon, but the generation of an interpreter was not yet
			implemented. 
			\todo{mention MSOS etc?}
			\todo{interpret is incorrect word, should check MSOS papers for
			better term}
		

	\subsection{\Rascal}
		The \Rascal funcons implementation aims to closely match the \ASFSDF
		implementation.
		\Rascal however, has more features suited for this domain and we have
		chosen to use those features to showcase the possible advantages.

		The first stage is similar, \Rascal also uses \CSF to generate the
		funcon specifications.
		However, since \Rascal has functions and \ADTs we do not generate 
		funcons as language specifications but we generate \Rascal files containing
		funcons as a \ADT structure.
		These funcons \ADTs can then be used to define the semantics of a
		programming language in a similar fashion as in \ASFSDF.

		\paragraph{Current limitations:}
			Same as in \ASFSDF.


\section{Manual implementation}
	We first created a manual translation of the \SDF funcons,
	since only a subset of the funcons were defined in the \CSF specifications.
	While this translation lacks modularity, it does provide an insight of how
	a \Rascal funcons implementation could work.

	\subsection{The funcons \ADT}
		The \filename{src/funcons/funcons.rsc} file contains the complete funcons
		\ADT equivalent to the \SDF implementation\footnote{Since the minus
		character is not allowed in \Rascal names, we used camel casing.}.
		Below is a snippet from this file, which shows the \CSF \code{Expr}
		sort.

\begin{rascal}
data Expr 
	= \data(Data dt)
	| abs(Patt pattern, Comm command)
	| application(Op op, list[Expr] exprs)
	| assignVar(Expr var, Expr val)
	| bound(Id id)
	| deref(Expr var)
	| derefIfVar(Expr var)
	| env(Decl decl)
	| newVar(Expr typeExpr)
	| newVar(Type \type)
	| op(list[Expr] expressions)
	| tup(list[Expr] tuples)
	| tupSeq(list[Expr] tuples)
	| typed(Expr expr, Type check)
	;
\end{rascal}
		
		\subsubsection{Funcons in \code{Op}}
		The funcons in \code{Op} causes the biggest difference in implementation, since
		the \SDF definition creates a structure not easily represented in the \ADT
		structure of \Rascal (these funcons are an example of funcons only defined
		in \SDF).

		Take for example the funcon \code{int-plus} which can be used used as 
		\code{int-plus(	Expr[[EXP1]], Expr[[EXP1]])}, 
		but also as 
		\code{comp(int-plus, int-neq)}.

		We could solve this by adding the \code{int-plus} (and all the others)
		to the \code{Data} and \code{Expr} sorts and the \code{Op} sort, but this would
		pollute our \ADT definition.
		So we chose to only add \code{int-plus} to the \code{Op} sort and add a
		\code{application} alternative to the \code{Data} and \code{Expr} sorts.
		This would mean we would have to write the first example as:
		\irascal{application(intPlus(), [Expr(EXP1), Expr(EXP2)])}.

		However, \Rascal's rewrite rules allows us to write 
		\irascal{intPlus(Expr(EXP1), Expr(EXP2))}.
		It is important to note that we assume to know the amount of parameter
		used by the funcon, else we would have to wrap the parameters
		into a list.

		\subsubsection{Explicit production chains}
		In the \ASFSDF implementation, production chains are often used.
		This is not visible for the funcon users since \ASFSDF allows implicit
		chaining.
		\Rascal does not support this, which makes funcon usage
		as in \code{Expr[[ NatCon ]] = NatCon} become
		\code{\ldots\ = data(int(nat(Nat::natCon(val))))} in \Rascal.
		We can solve this with a rewrite rule, but it means that for each funcon
		we have to determine where it is actually used and create a rewrite rule
		for that usage.
		We expect this has a fairly simple solution.
	
	\subsection{Semantic description using funcons}

		The \filename{src/lang/pico/semantic/Main.rsc} file contains the semantic
		description of the pico language, similar to the semantic description of
		the pico language in \ASFSDF.
		Below are two definitions, the first is for the loop command, and the
		second is for the add expression.

\begin{rascal}
public Comm pico2Comm(Statement::loop(con, body)) =
	whileLoop( 
		noteq(pico2Expr(con), natCon(0)), 
		pico2Comm(body)
	);

public Expr pico2Expr(add(lhs, rhs)) = intPlus(pico2Expr(lhs), pico2Expr(rhs));
\end{rascal}

		\subsubsection{Naming collisions}
		In the example we see that the patterns in the parameters of a function use
		the qualified name for a constructor, this is due to naming collisions
		with the funcon names.
		The \AST contains the \code{loop} alternative for the \code{Statement}
		non-terminal, but the \code{Comm} sort also defines the \code{loop} funcon.
		In most places \Rascal correctly handles these collisions, but at the
		moment the patterns require fully qualified names to avoid these collisions.
		We expect these issues to be solved when the static type checker is
		integrated.

		\subsubsection{\AST instead of \CST}
		For our implementation we chose to write the semantic definition against
		the \AST of the pico language, instead of the \CST as the \ASFSDF
		solution does.
		However, this is not due to a limitation of \Rascal, below is a snippet
		of how the \code{pico2Expr(add(lhs, rhs))} would look if written for the
		\CST.

\begin{rascal}
public Expr pico2Expr(`<Expression lhs> + <Expression rhs>`) 
	= intPlus(pico2Expr(lhs), pico2Expr(rhs));
\end{rascal}

\section{CSF generated implementation}
	We have demonstrated that all the funcon concepts from \ASFSDF can be
	translated to \Rascal in the previous section.
	We will now explain how we could use \Rascal to generate the funcons from
	the \CSF specifications.

	\subsection{\CSF}
	The \CSF file defines the funcon, it describes it in natural language and it
	formalizes its semantics using \MSOS style notation.
	Below is the definition of the \code{decl} funcon.
	We see that the funcon is part of the \code{Decl} sort and takes a
	\code{Comm} as parameter.
	
\begin{lstlisting}[language=CSF]
Decl ::= decl(Comm)

Glossary:

This allows $Comm$ to be executed in a sequence of declarations,
computing the empty environment.

Alias:  declaration

1:      Comm --> Comm'
        ---
        decl(Comm) --> decl(Comm')

2:      decl(skip) --> map-empty

3:      Comm ==> Comm'
        ---
        decl(Comm) ==> decl(Comm') : map-empty

\end{lstlisting}

	We rewrote the \SDF grammar definition in \Rascal, this grammar can be found in the
	\filename{src/csf2rascal/lang/csf/cst/MainCSFGrammar.rsc} file.
	The only real difference is that we marked the words such as ``Glossary'' as
	keywords, this is important for disambiguation and provides the \IDE with
	information for simple highlighting.

	Since \Rascal provides the \code{implode} function which automatically
	transforms a \CST onto a \AST, we defined the \AST for \CSF and used that \AST to
	write our generator.
	The \AST is defined in the \filename{src/csf2rascal/lang/csf/ast/Main.rsc} file.

	\subsection{\Rascal funcons generation}
	Similar to the \ASFSDF approach (\filename{CSF-to-SDF.asf}) we generate
	\Rascal files using \Rascal.
	The \filename{src/csf2rascal/lang/csf/generation/Generate.rsc} file
	implements this generation in 249 \LOC (blank lines excluded).

	Unlike the manual implementation, the generated funcons have the same
	modularity as the \ASFSDF implementation.
	A funcon user can import only the needed funcons.
	We have added a two small `features' compared to the \ASFSDF generation.
	We added support for the aliased funcons, and we used \Rascal's
	\code{@doc\{\}} source code annotations to add the informal description of a
	funcon.

	Below is the \Rascal implementation of the above \CSF file.

\begin{rascalModule}
module csf2rascal::generated::Decl::decl_Comm::decl

extend csf2rascal::generated::Decl::Decl;
import csf2rascal::generated::Comm::Comm;
 
@doc{This allows Comm to be executed in a sequence of declarations, computing the
 empty environment.}
data Decl = decl(Comm comm);

public Decl declaration(Comm comm) = 
	decl(comm);
\end{rascalModule}
	
	\paragraph{Limitations:}
	We have not solved the problems caused by the explicit production chains
	since we cannot translate all the funcons automatically.
	Moreover, since the \code{Op} sort was not described in a \CSF
	specification, we have not generated rewrite rules to allow easier usage of
	those funcons.

	\subsection{Semantic description using generated funcons}
	We have no semantic description of the Pico language using these funcons,
	since we cannot generate all the funcons from \CSF files.
	Moreover, when manually comparing the generated \ADT with the manual \ADT we
	see that they are the same.
	Therefore the semantic description would be a duplication of the earlier
	manual effort.

\section{Limitations of \Rascal implementation}
This section will provide a list of the limitations we found while implementing
the funcons in \Rascal.

\begin{enumerate}
	\newcommand{\limit}[2]{\item \textbf{#1,} #2}
	\limit
		{No implicit chains in the \ADTs}
		{
			this imposes either a challenge for the \CSF to \Rascal generator or
			the user of the funcons.
			We suggest that with some smart work in the \Rascal generator this
			problem can be solved.
		}

	\limit
		{Naming collisions}
		{
			in the current version some naming collisions can cause confusions
			for the funcon user.
			We aim to improve these naming collisions and provide better error
			messages.
			These efforts are related to our project for finishing the static type
			checker.
		}
	
\end{enumerate}

\section{Advantages of \Rascal implementation}

This section will describe advantages we found and envision when implementing
the funcons in \Rascal.

\begin{enumerate}
	\newcommand{\advantage}[2]{\item \textbf{#1,} #2}

	\advantage
		{Simpler generator}
		{
			since one of \Rascal's aims is to support code generation, there are
			more concepts built into the language than available in \ASFSDF.
			Moreover, \Rascal has native support for regular
			expressions and has an extensive library for common string, set,
			list, and map operations.
		}

	\advantage
		{Simple \CST to \AST transformation}
		{
			to avoid repeating the syntactical concepts of \CSF, or the language
			parsed, \Rascal offers a simple way to implode the \CST to a \AST.
			This create better maintainable code without the extra maintenance
			of this transformation.
		}

	\advantage
		{Eclipse \IDE}
		{
			\Rascal takes advantage of many features of the Eclipse \IDE, this
			implies that it is easy to provide the users of funcons with a rich
			user experience, such as syntax highlighting and outlining.
		}

	\advantage
		{Writing the interpreter in \Rascal}
		{
			it would be possible to write the funcons interpreter using \Rascal.
			For a example of the possibilities see the
			\filename{std/library/demo/lang/MissGrant/} folder.
		}

	\advantage
		{Writing a compiler in \Rascal}
		{
			it would also be possible to write a funcons compiler using \Rascal.
			For a example of the possibilities see the
			\filename{std/library/demo/lang/MissGrant/} folder or check our
			Oberon0
			compiler\footnote{\url{http://svn.rascal-mpl.org/oberon0/trunk/}}
			which compiles to Java, \JVM bytecode, and C.
		}
\end{enumerate}
\end{document}
