\RequirePackage[l2tabu,orthodox]{nag}
\documentclass[a4paper, oneside, notitlepage]{article}

\usepackage[english]{babel}
\usepackage{fullpage}
\usepackage{fixltx2e}
\usepackage{xspace}
\usepackage{relsize}

% font stuff
\usepackage[T1]{fontenc}
\usepackage{microtype}
\usepackage{lmodern} \normalfont
% lmodern has no smallcaps bold , so we replace it by computer-modern (closely
% related)
\DeclareFontShape{T1}{lmr}{bx}{sc} { <-> ssub * cmr/bx/sc }{}

\title{Funcons in Rascal - report}
\author{Davy Landman}

\newcommand{\Rascal}{\textsc{Rascal}\xspace}
\newcommand{\CSF}{\textsmaller{CSF}\xspace}
\newcommand{\ASFSDF}{\textsmaller{ASF-SDF}\xspace}
\newcommand{\SDF}{\textsmaller{SDF}\xspace}
\newcommand{\ASF}{\textsmaller{ASF}\xspace}
\newcommand{\AST}{\textsmaller{AST}\xspace}
\newcommand{\CST}{\textsmaller{CST}\xspace}
\newcommand{\ADT}{\textsmaller{ADT}\xspace}
\newcommand{\ADTs}{\textsmaller{ADTs}\xspace}

	
\begin{document}
\maketitle

\section{The architecture}
	\subsection{\ASFSDF}
		In \ASFSDF the funcons are implemented in a two stage approach.
		The first stage is `compiling' the \CSF~(funcons specifications) into
		\SDF definitions.
		These definitions can then be used to describe the semantics of a
		programming language.
		These funcon semantics are connected to syntax of the programming
		language using the \ASF equations.

		\paragraph{Current limitations:}
			At the time of writing, not all funcons are specified in \CSF,
			which means that not all the \SDF defined funcons were generated
			from the \CSF specifications. 
			Moreover, the \CSF specifications contain the descriptions of how to
			`interpret' a funcon, but the generation of an interpreter was not yet
			implemented. 
		

	\subsection{\Rascal}
		The \Rascal funcons implementation aims to closely match the \ASFSDF
		implementation.
		\Rascal however, has more features suited for this domain and we have
		chosen to use those features to showcase the possible improvements.

		The first stage is similiar, \Rascal also uses \CSF to generate the
		funcon specifications.
		However, since \Rascal features functions and \ADTs we do not generate 
		funcon as a language specification but we generate \Rascal files containing
		the funcon \ADT structure.
		These funcon \ADTs can than be used to define the semantics of a
		programming language in a similar fashion as in \ASFSDF.

		\paragraph{Current limitations:}
			Same as in \ASFSDF.


\section{Manual implementation}

\section{CSF generated implementation}

\section{Limitations of rascal implementation}

\section{Advantages of rascal implementaton}

\end{document}
